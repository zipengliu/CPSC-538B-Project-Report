
\section{Abstract}
\label{sec:abstract}
%%%%%%%%%%%%%%%%%%%%%%%%%%%%%%%%%%%%%%%%%%%%%


Understanding distributed systems is difficult. Concurrency and
non-determinism leads developers to misinterpret their systems
behaviour. Indeed, even small systems require complex reasoning to
determine if their behaviour is correct.  Developers
characteristically log executions; when unspecified or deviant
behaviour is observed, the culprit is found by laboriously inspecting
logs. Specifications are useful for defining correct behaviour, and
helping developers understand their systems abstractly, but real
systems often drift substantially from their specifications. Tools
which mine specifications from real systems can help developers
understand their systems at the same level of abstraction, and quickly
identify deviant behaviour.  Distributed state, communication
patterns, and data invariants are useful execution artifacts for
reasoning about a systems behaviour.  However, interpreting them from
logs alone is tricky and error prone.  Using visualizations, large
amounts of information can be concisely, and meaningfully encoded.  We
propose Pangaea, a visual tool to understand execution traces in
distributed systems.  Using novel distributed state differentiation
Pangaea builds approximate, high level, finite state machines derived
from dynamic behaviour. Deviations from expected state machines can be
inspected using a graph of the systems invariants, and communication
graph. These visualization provide a means for identifing deviant
behaviour at an abstract level, and facilities for zooming in on
concrete details.


\section{Abstract}
\label{sec:abstract}
%%%%%%%%%%%%%%%%%%%%%%%%%%%%%%%%%%%%%%%%%%%%%


Understanding distributed systems is difficult. Concurrency and
non-determinism leads developers to misinterpret their systems
behaviour. Indeed, even small systems require complex reasoning to
determine if their behaviour is correct.  Developers
characteristically log executions; when unspecified or deviant
behaviour is observed, the culprit is determined through laborious log
inspection. Specifications are useful for defining correct behaviour,
but proving that a system adheres to a specification is beyond the
capabilities of modern techniques.  Such shortcomings demand new tools
to help developers better understand their systems, and quickly
identify deviant behaviour.  
Distributed
state, communication patterns, and data invariants are useful
execution artifacts for reasoning about a systems behaviour.  However,
interpreting them from logs alone is tricky and error prone.  
Using visualizations, large amounts of
information can be concisely, and meaningfully encoded. 
We propose Pangaea, a visual tool to understand execution traces in
distributed systems.  
Using novel distributed state differentiation
Pangaea builds approximate, high level, finite state machines derived
from dynamic behaviour. Deviations from expected state machines can be
inspected using a graph of the systems invariants, and communication
graph. 
%\textbf{We evaluated Pangaea$\dots$}.

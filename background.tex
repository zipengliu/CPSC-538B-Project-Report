\section{Background}
\label{sec:background}


\noindent{\textbf{Distributed Snapshot}} is an algorithm which
captures consistent distributed state without interfering with the
execution of a system itself~\cite{dist_snapshots_Chandy1985}.
Distributed snapshots can be computed online or mined from a log
containing vector clocks which provide a partial ordering of events in
a system~\cite{mattern_vector_clocks_1989}. We consider distributed
snapshots to be a fundamental granularity for state in a distributed
system. Our state analysis technique is therefore applied at the level
of a distributed snapshot.

\noindent{\textbf{\dinv}} is a tool which detects likely data invariants in distributed
systems~\cite{dinv}. \dinv operates by instrumenting distributed
systems to log state and vector clocks. Execution logs from the nodes
of the system are merged together. The state of the system is
reconstructed and output as a distributed system trace. \dinv
leverages Daikon to automatically infer data invariants on the trace~\cite{Ernst01}.
We use \dinv as a tool for capturing distributed state.

\noindent{\textbf{Visualization}} is useful for helping users
comprehend complicated data.  Understanding distributed systems is a
challenging task for system designers, software developers and
students because of their inherent complexity.  Prior work on the
visualization on system traces has demonstrated that similarities in
the traces between software versions can be meaningfully conveyed to
developers~\cite{6613833}~\cite{Reynolds_detectingthe}. Alternative
work has demonstrated that visualizing concurrent system traces using
partial orderings can help developers reason about interleaving
executions~\cite{6650534}~\cite{7272586}~\cite{isaacs2014combing}.  We
consider the state of a system to be a direct artifact of a system
trace.  To our knowledge few attempts have been made to visualize
distributed system traces, and by extension distributed state traces.
Beschastnikh et. al built the ShiViz~\cite{BeschastnikhWBE2016} to
generate an interactive communication graph (a time-space diagram)
using distributed system execution logs, which is a stream of events
with their vector clocks.  The happened-before relation between events
is a property of vector clocks, and it establishes a partial ordering
on all events in a system.  Such a communication graph is
straightforward to understand who sends what to whom at a specific
time, but not easy to grasp the big picture, especially distributed
states.  Though ShiViz can discover graph ``motifs'', frequently
recurring communication patterns, but it is limited to work under 4
events and certain motif finding algorithms that might not understand
the communication pattern as a person does.  Isaacs et. al developed
Ravel~\cite{isaacs2014combing} to visualize parallel execution traces,
but their focus is on how to scale it up in the number of nodes and
span of time, which actually ignores the state of the system.  Ravel
can provide insights for certain communication patterns, but it cannot
know any information about important distributed states.

%There are a few studies focusing on 
%performance diagnositics of parallel or distributed systems, but few are done
%to understand the distributed nature in terms of state machines: 
%how distributed states transit.
%Visualization is suitable when there is enough data but no automatic algorithms
%to dig out the insights that people are looking for.  Here we can generate logs
%that we need by \textbf{TODO}, and there does not exist a general model that 
%can capture the distributed nature.

%\begin{itemize}
%\item modular visualization of distributed systems
%\item pip \textit{http://issg.cs.duke.edu/pip/}
%\item Overview: A Framework for Generic Online Visualization of Distributed
%Systems
%\item Jumpshot \textit{http://www.mcs.anl.gov/research/projects/perfvis/software/viewers/}
%\item vampire: http://citeseer.ist.psu.edu/viewdoc/summary?doi=10.1.1.38.1615
%\end{itemize}

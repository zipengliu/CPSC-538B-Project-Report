%%%%%%%%%%%%%%%%%%%%%%%%%%%%%%%%%%%%%%%%%%%%%
\section{Discussion}
\label{sec:discussion}
%%%%%%%%%%%%%%%%%%%%%%%%%%%%%%%%%%%%%%%%%%%%%


A course project is limited by the scope, but there is a lot more can be done to make
Pangaea a more practical tool instead of a prototype. 
First of all, there are some additional tasks we can implement to enhance the 
functionality of Pangaea.  Currently users have no control over the differentiation
function, but to mitigate various distributed systems, we might require various 
differentiation functions.  We can switch between multiple ways of calculation of 
two states, allow users to control what should be consider into the distance, and 
also composition of both.
For example, we can have a subtraction function for variables with the type of a number,
xor function for other types, and average all the differentiation parts.
We need an interface to specify these differentiation algorithms.

Second, it is easy to see the big picture from the time curve, but hard
to understand the cause of proximity (why two states are close or distant).  This is
actually a known problem for dimension reduction: despite the fact that people
can see clusters from the result of dimension reduction, but they have problems in
understanding and trusting the results, especially there is non-linear deformation.
There is already some effort to address this problem~\cite{stahnke2016probing}, but
in our case, we can mitigate by allowing users to compare two or multiple states
with visual aids, such as encoding the largest difference between two selected states.
Addressing comparison among states would allow users to better understand 
the details of transition of states.


Third, the invariants graph only takes into account the output of Daikon.  
Imagine there is a bug in the execution which breaks an assumed invariant.  Daikon
would not output anything about this because the assumed invariant is 
invalidated only once.  There is no way the user can detect such kind of infrequent
anomalies because it is absent from the input of the invariant graph in the first
place.  Therefore, it would be beneficial to also consider the input of Daikon, which
is, actually, the dumped variables for each state.  For example, for bi-variable
relationships, we can use scatterplot matrix (SPLOM) with techniques like scagnostics
~\cite{wilkinson2006high} to narrow down interesting areas.

To make a practical tool, we also need usability evaluations to make sure users can
understand the graphical representation and ways of interaction.
%%%%%%%%%%%%%%%%%%%%%%%%%%%%%%%%%%%%%%%%%%%%%
\section{Introduction and Motivation}
\label{sec:intro}
%%%%%%%%%%%%%%%%%%%%%%%%%%%%%%%%%%%%%%%%%%%%%


\sg{I think that the goal of this paper should be to convince readers that Distributed systems require incremental approaches to understanding them}
\begin{itemize}
    \item \textbf{Understanding}

        Understanding distributed systems is difficult. Concurrency
        and non-determinism leads developers to misinterpret their
        systems behaviour. Indeed, even small systems require complex
        reasoning to determine if their behaviour is correct.

    \item \textbf{Current Approaches}
        
        Developers characteristically log executions; when unspecified or
        deviant behaviour is observed, the culprit is determined
        through laborious log inspection. Specifications are useful for
        defining correct behaviour, but proving that a system adheres
        to a specification is beyond the capabilities of modern
        techniques.


    \item \textbf{Lack of tools}

        Such shortcomings demand new tools to help developers
        better understand their systems, and quickly identify deviant
        behaviour.

    \item \textbf{Visualization}

        Using visualizations, large amounts of information can be
        concisely, and meaningfully encoded. Distributed state,
        communication patterns, and data invariants are useful
        execution artifacts for reasoning about a systems behaviour.
        However, interpreting them from logs alone is tricky and error
        prone.

    \item \textbf{Contributions}

        We propose Pangaea, a tool for identifying deviant behaviour
        in distributed systems. Using novel distributed state
        differentiation Pangaea builds approximate, high level, finite
        state machines derived from dynamic behaviour. Deviations
        from expected state machines can be inspected using a graph of
        the systems invariants, and communication graph.

    \item \textbf{Eval}
        We evaluated Pangaea \sg{will we though}
\end{itemize}


Developing distributed systems is a difficult task. Inherent concurrency,
non-determinism, complicate understanding how a system behaves. Developers
lack tool for providing insight about the state of a system during it's
execution.  The lack of insight makes triaging bugs an arduous task involving
the manual inspection of multiple logs. Visualization is useful for quickly
articulating information. Currently there are no tools which visualize the
state of a distributed systems, written in Go, to aid developers. We propose \dviz a tool
for visualizing the state of arbitrary distributed systems. \dviz profiles a
systems execution by capturing a systems state and rendering features of its
state transitions. \dviz produces predicable patterns for systems executing
under normal conditions, and alternative patterns for unusual executions.
Developers can use irregularities in \dviz's output to quickly triage buggy
behaviour.

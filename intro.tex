%%%%%%%%%%%%%%%%%%%%%%%%%%%%%%%%%%%%%%%%%%%%%
\section{Introduction and Motivation}
\label{sec:intro}
%%%%%%%%%%%%%%%%%%%%%%%%%%%%%%%%%%%%%%%%%%%%%

The behaviour of distributed systems is often complicated and
difficult to understand. Even small simple systems require that
developers reason about concurrancy, and the possibilty of
communication and node failure. In order to develop correct systems
and triage bugs developers must construct a menatal model for how
their system should behave. Typically distributed systems are built by
first strictly specifing the system, then developing the system to the
specification. This approach has the advantange that the specification
can be verified, and that it helps developers build mental models of
their systems prior to writing
code~\cite{Newcombe:2015:AWS:2749359.2699417,WilcoxWPTWEA2015}.
Distributed system specification are simplified versions of the
software which define finite state machines (FSM), and invariants
based on protocol specific behaviour. FSM are much easier to reason
about than source code, and provide developers an intuitive
understanding of how their system should behave in all cases.
Invariants are likewise useful in that they constrain the behaviour of
a system which shrinks the space of behaviours developers need to
reason about. However, specifications are not source code. As
implementations grow they drift further from their abstract
specifications~\cite{917525}. When bugs occur developers typically use
a systems logs to triage the problem. These logs can be massive, and
identifiy buggy behaviour is a labourious error prone task.  In order
to understand their real systems developers require tools which can
communicate complicated concurrent behaviour. Tracing tools such as
Dapper~\cite{36356}, and Lprof~\cite{Zhao:2014:LNR:2685048.2685099}
are usefull for reasoning about specific behavior, but do little to
reinforce developers mental models. ShiViz~\cite{BeschastnikhWBE2016}
provisions users with a high level visual of a systems communication,
and message logs, but does not provide a mechanism for reasoning about
a nodes state. We propose Pangaea a tool for identifing deviant
behaviour in distributed systems. Pangaea uses novel techniques for
building a time curve, or aproximate FSM, from the logs of a system.
Further, Pangaea encorportes a graphicial representation of likely
data invariants, and a ShiViz style communciation graph. These
features allow developers to quickly identify deviant behaviour in
their systems by contrasting their mental model with state machines
and invariants detected at runtime. When deviant behavior is detected
Pangaea provides tools for zooming in on the execution, to examine
concrete state.



\textbf{We evaluated Pangaea$\dots$}.

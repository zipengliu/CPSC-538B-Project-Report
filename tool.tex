%%%%%%%%%%%%%%%%%%%%%%%%%%%%%%%%%%%%%
\section{Tool}
\label{sec:tool}
%%%%%%%%%%%%%%%%%%%%%%%%%%%%%%%%%%%%%

Pangaea is implemented in 3 components, a state differenitation
component \footnote{https://github.com/wantonsolutions/Dviz}, a server
for hosing pre-computed examples
\footnote{https://github.com/zipengliu/PangaeaServer}, and a client
which renders each visual component
\footnote{https://github.com/zipengliu/Pangaea} .

The state differentiation component is written in Go, and implemented
in 389 LOC. Our decisions to write our differential caluator is Go was
two fold. First Dinv analyzes systems writen in Go, and generated Gob
encoded output. By using Go we substantially reduced the effort of
integrating the two projects. Second, Go is fast, and has powerful
language support for concurrancy. Calculating snapshot differentials
is an imbarisingly parallel task, and we were able to produce a
concise automatically scaleable solution.

Pangaea's client and server are written in Javascript, and use
Node.js. Additionally the client is written using the React framework.
The client is 1597 LOC, and the server is 140. We chose Javascript to
make Pangaea easily accessable to users. The time and effort of
insturmenting, and anyalzing a distributed system with Dinv is too
large for most users, and we wanted to provide them with a means to
quickly interact with Pangaea's visualizations.


